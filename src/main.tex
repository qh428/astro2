\documentclass[a4paper,11pt]{article}

% Packages
\usepackage{amsmath}
\usepackage{amssymb}
\usepackage[ngerman,english]{babel}

\usepackage{microtype}
\usepackage{siunitx}
\usepackage{hyperref}
\usepackage{geometry}
\usepackage{enumitem}
\usepackage{MnSymbol}

\usepackage{longtable}
\usepackage{threeparttable}
\usepackage{threeparttablex}
\usepackage{booktabs}
\usepackage{multirow}
\usepackage{multicol}

% Document-specifics
\author{Nils Hoyer}
\date{\today}
\title{%
  Referenzdokument für die Vorlesung\\ ``Einführung in die Astronomie 2''\\[10pt]
  \large
  Sommersemester 2022
}

\newcommand{\entry}[4]{%
\begin{tabular}{l}%
#1\\
\enspace(engl.\ \textit{#2})\\
\end{tabular}
& #3 & #4\\\addlinespace}

\newcommand{\dm}{dm}


% Package-specifics
\geometry{%
  a4paper,
  top=60pt,
  left=60pt,
  right=60pt,
  bottom=120pt
}

\hypersetup{%
  draft=false,
  colorlinks=true,
  linkcolor=blue,
  anchorcolor=blue,
  citecolor=blue,
  filecolor=blue,
  menucolor=blue,
  runcolor=blue,
  urlcolor=blue,
  pagebackref=true,
  bookmarks=true,
  linktoc=page,
  pdftitle={Referenzdokument, Astro2},
  pdfauthor={Nils Hoyer},
  pdfcreator={xelatex},
  pdfproducer={XeLaTeX}
}

\DeclareSIUnit \pc {pc}
\DeclareSIUnit \erg {erg}
\DeclareSIUnit \mag {mag}
\DeclareSIUnit \arcsec {arcsec}
\DeclareSIUnit \Lsun {L_{\odot}}
\DeclareSIUnit \Msun {M_{\odot}}



\begin{document}

\selectlanguage{ngerman}
\pagenumbering{gobble}
\thispagestyle{empty}
\maketitle


\begin{abstract}
Dieses Dokument stellt eine Zusammenfassung der, aus Tutorensicht, wichtigsten Formeln, Umrechnungen, und Konstanten dar.
Ziel ist eine relativ kompakte Uebersicht ueber die relevantesten Gleichungen und bekommen, wobei besonderer Wert auf eine korrekte Notation gelegt wird.
Leider geht besonders der letzte Teil in den meisten Vorlesungen und einigen Publikationen unter, sodass dieses Dokument die Relation von Parametern zueinander verdeutlichen soll.

Aus hoffentlich offensichtlichen Gruenden ist dieses Dokument nicht vollstaendig.
Grund hierfuer ist der zeitliche Aufwand versus die verfuegbarer Zeit, die verwendete Literatur (falls zutreffend), aber auch das limitierte Wissen der Tutoren.
Dem entsprechend behalten sich die Autoren vor, Ergaenzungen oder Aenderungen durchzufuehren.
Am Ende dieser Seite wird jedoch das Datum der letzten Aenderung und die Version des Dokuments angegeben, falls Sie mit Ihrem lokalen Dokument vergleichen wollen.

Anregungen, Kritik, oder Ergaenzungen sind erwuenscht und jederzeit willkommen.

\vspace{10pt}
\selectlanguage{english}
For the time being, as this document is related to a lecture given in German, the document will not be available in English language.
However, keywords will be provided both in German and English, thus, non-German speaker should be able to follow along.
In the future, an English version may be available.
\selectlanguage{ngerman}
\end{abstract}


\vfill

Letzte Aenderung vom \today.
\hfill
Versionsnummer: 0.2


\clearpage
\section{Notation}
\label{sec:notation}

Zunächst werden wir die in diesem Dokument verwendeten Symbole aufzählen.


\begin{ThreePartTable}

  \renewcommand\TPTminimum{\textwidth}

  %\begin{TableNotes}
  %  \item[(a)] My item
  %\end{TableNotes}

  \begin{longtable}{ l @{\extracolsep{\fill}} *{3}{l}}
    \caption[]{Notation für dieses Dokument}\\
    \label{tab:notation}\\
    \toprule
    Begriff & Notation & Einheit\\
    \midrule
    \endfirsthead

    \toprule
    Begriff & Notation & Einheit\\
    \midrule
    \endhead

    \midrule[\heavyrulewidth]
    \multicolumn{3}{l}{\textit{fortgesetzt}}\\
    \endfoot

    \midrule[\heavyrulewidth]
    %\insertTableNotes
    \endlastfoot

    \entry{Fluss}{flux}{$F$}{[\si{\erg\per\centi\metre\squared\per\second}]}
    \entry{Distanz}{distance}{$d$}{[\si{\centi\metre}] oder vergleichbar}
    \entry{Scheinbare Helligkeit}{apparent magnitude}{$m$}{[\si{\mag}]}
    \entry{Absolute Helligkeit}{absolute magnitude}{$M$}{[\si{\mag}]}
    \entry{Distanzmodul}{distance modulus}{\dm}{[\si{\mag}]}
    \entry{Oberflächenheligkeit}{surface brightness}{$\mu$}{[\si{\mag\per\arcsec\squared}], [\si{\Lsun\per\pc\squared}], [\si{\Msun\per\pc\squared}]}
    \entry{Rotverschiebung}{redshift}{$z$}{N/A}

  \end{longtable}
\end{ThreePartTable}

\clearpage
\section{Formeln und Umrechnungen}
\label{sec:formeln_und_umrechnungen}

Dieser Abschnitt wird in drei Unterkategorien aufgeteilt: Photometrie / Spektroskopie, Kinematik und Verschiedenes.
Zunächste wiederholen wir ein paar grundlegende Formeln und Umrechnungen aus der Vorlesung ``Einführung in die Astronomie 1''.


\subsection{Photometrie \& Spektroskopie}
\label{subsec:photometrie_und_spektroskopie}

\begin{enumerate}[label=$\smalltriangleright$]
  \item
  Die Intensitaet $I$ (engl.\ \textit{intensity}) gibt eine Energie pro Flaeche, Zeit und Winkel an.
  Der Strahlungsfluss $F$ (engl.\ \textit{flux}) haengt mit der Inentistaet ueber ein Oberflaechenintegral zusammen:
  \begin{equation}
    F = \int_{S} \mathrm{d}S \; I = 4\pi I \quad ,
  \end{equation}
  wobei wir in fuer das zweite Gleichheitszeichen eine einfach sphaerische Anordnung annehmen.
  In der Literatur findet man oftmals eine Defition ohne den Raumwinkelfaktor $4\pi$.


  \item
  Der Strahlungsfluss $F$ (engl.\ \textit{flux}) gibt die eingehende Energie pro Fläche und Zeit an.
  Üblicherweise hat $F$ die Einheit [\si{\erg\per\centi\metre\squared\per\second}].


  \item
  Ein Schwarzer Strahler (engl.\ \textit{black body}) ist ein idealisierter Körper, welcher jede einfallende Strahlung absorbiert.
  Er emittiert Strahlung, die der Planck'schen Funktion (engl.\ \textit{Planck's law}) folgt:
  \begin{equation}
    B_{\nu, T} = 2 \Big( \frac{\nu}{\mathrm{c}} \Big)^{2} \frac{\mathrm{h}\nu}{\exp \big( \frac{\mathrm{h}\nu}{k_{\mathrm{B}} T} \big) - 1} \quad,
  \end{equation}
  mit den Variablen $\nu$ als Frequenz und $T$ als Temperatur.
  $\mathrm{h}$ ist das Planck'sche Wirkungsquantum, $\mathrm{c}$ die Lichtgeschwindigkeit, und $k_{\mathrm{B}}$ die Boltzmannkonstante.


  \item
  Der Strahlungsfluss $F$ eines Schwarzen Körpers hängt mit seiner Oberflächentemperatur $T$ zusammen.
  Es gilt
  \begin{equation*}
    F = \sigma_{\mathrm{SB}} T^{4} \quad ,
  \end{equation*}
  wobei $\sigma_{\mathrm{SB}}$ als Stefan-Boltzmann Konstante (engl.\ \textit{Stefan-Boltzmann constant}).

  In der Astronomie nehmen wir an, dass Sterne ideale Schwarze Körper sind.
  Dies bedeutet, dass wir die Sterne durch eine effektive Temperatur $T_{\mathrm{eff}}$ beschrieben werden können.
  Diese Temperatur besitzt ein Schwarzer Körper, wenn er den selben Strahlungsfluss wie ein Stern emittiert.
  Daher erhalten wir eine Relation zwischen dem Fluss eines Sterns und dessen effektive Temperatur:
  \begin{equation}
    F = \sigma_{\mathrm{SB}} T_{\mathrm{eff}}^{4} \quad  \quad .
  \end{equation}


  \item
  Das Wien'sche Verschiebungsgesetz (engl.\ \textit{Wien's displacement law}) ist eine Approximation für das Maximum der oben aufgeführten Planck'schen Funktion.
  Es ist gegeben durch
  \begin{align}
    \nu_{\mathrm{max}} \; [\si{\hertz}] & \approx 3 \frac{k_{\mathrm{B}} T}{\mathrm{h}} \\
                                     & \approx \frac{\SI{5.879e10}{\hertz\per\kelvin}}{T} \quad .
  \end{align}

  Man kann das Verschiebungsgesetz auch also Funktion der Wellenlänge $\lambda$ ausdrücken:
  \begin{equation}
    \lambda_{\mathrm{max}} \; [\si{\micro\metre}] \approx \frac{\SI{2898}{\micro\metre\per\kelvin}}{T} \quad .
  \end{equation}


  \item
  Die scheinbare Helligkeit $m$ (engl.\ \textit{apparent magnitude}) wird durch einen Strahlungsfluss $F$ bestimmt.
  Es gilt
  \begin{equation}
    m = -2.5 \log_{10} \bigg( \frac{F}{F_{\mathrm{ref}}} \bigg) \quad ,
  \end{equation}
  wobei $F_{\mathrm{ref}}$ ein Referenzfluss ist.
  Typischerweise nutzt man für $F_{\mathrm{ref}}$ den Fluss des Sterns Vega, jedoch existieren auch andere Referenzflüsse.


  \item
  Die absolute Helligkeit $M$ (engl.\ \textit{absolute magnitude}) entspricht der scheinbaren Helligkeit bei einer Distanz von $d = \SI{10}{\pc}$.
  Daher gilt
  \begin{equation}
    M = m - 5 \log_{10} \frac{d}{[\SI{10}{\pc}]} \quad .
  \end{equation}


  \item
  Der Distanzmodul {\dm} (engl.\ \textit{distance module}) entspricht der Different der scheinbaren und absoluten Helligkeit und kann zur Distanzbestimmung verwendet werden:
  \begin{equation}
    \mathrm{\dm} = m - M = 5\log_{10} \bigg( \frac{d [\si{\pc}]}{\SI{10}{\pc}} \bigg) = 5 \log_{10} ( d [\si{\pc}]) - 5 \quad .
  \end{equation}


  \item
  Die Leuchtkraft $L$ (engl.\ \textit{luminosity}) eines Objektes kann direkt über die absolute Helligkeit bestimmt werden:
  \begin{equation}
    M - M_{\mathrm{ref}} = -2.5 \log_{10} \bigg( \frac{L}{L_{\mathrm{ref}}} \bigg) \quad \leftrightarrow \quad \frac{L}{L_{\mathrm{ref}}} = 10^{-0.4 (M - M_{\mathrm{ref}})} \quad ,
  \end{equation}
  wobei $M_{\mathrm{ref}}$ eine Referenzhelligkeit und $L_{\mathrm{ref}}$ eine Referenzleuchtkraft sind.
  Häufig verwendet man die absolute Helligkeit der Sonne $\mathrm{M}_{\odot}$, um eine Leuchtkraft in Einheiten der solaren Leuchtkraft $\mathrm{L}_{\odot}$ zu erhalten.


  \item
  Wir erhalten auch eine Leuchtkraft $L$, wenn wir den gesamten Fluss $F$ über eine Fläche $A$ integrieren:
  \begin{equation*}
    L = \int_{S} \mathrm{d}S \; F(\vec{x}) \quad .
  \end{equation*}
  Oftmals wird der Fluss als isotrop angenommen, sodass wir ein einfaches Oberflächeningetral erhalten.
  Für eine sphärische Symmetrie mit einem Radius $r$ erhalten wir
  \begin{equation}
    L = 4\pi r^{2} F = 4\pi r^{2} \sigma_{\mathrm{SB}} T_{\mathrm{eff}}^{4} \quad ,
  \end{equation}
  wobei wir beim letzten Gleichheitszeichen angenommen haben, dass wir das System als Schwarzen Körper approximieren können (z.B. ein Stern).


  \item
  Die Oberflächenhelligkeit $\mu$ (engl.\ \textit{surface brightness}) gibt den Fluss $F$ pro Fläche $A$ an:
  \begin{align}
    \mu \; [\si{\mag\per\arcsec\squared}] & = -2.5 \log_{10} \bigg( \frac{F}{F_{\mathrm{ref}}} \cdot \frac{1}{A \Big[ \si{\arcsec\squared} \Big] } \bigg) \\
                                          & = m + 2.5 \log_{10} \Big( A \Big[ \si{\arcsec\squared} \Big] \Big) \quad .
  \end{align}


  \item
  Über die weiter oben aufgeführten Formeln kann man folgende Formel errechnen:
  \begin{equation}
    \mu \; \Big[ \si{\Lsun\per\pc\squared} \Big] \approx \frac{1}{\alpha^{2} \; \Big[ \si{\arcsec\squared} \Big]} 10^{-0.4 \cdot (m \; [\si{\mag}] - \mathrm{M}_{\odot} - 21.572)} \quad ,
  \end{equation}
  wobei man eine Fläche $\alpha^{2}$ und eine Magnitude $m$ einsetzt, um eine Oberflächenhelligkeit $\mu$ zu erhalten.
  $\mathrm{M}_{\odot}$ ist die absolute Magnitude der Sonne.


  \item
  Falls bereits eine Oberflächenhelligkeit $\mu$ in der Einheit $\Big[ \si{\mag\per\arcsec\squared} \Big]$ gegeben ist, können wir diese natürlich in die Einheit $\Big[ \si{\Lsun\per\pc\squared} \Big]$ umrechnen:
  \begin{equation}
    \mu \; \Big[ \si{\mag\per\arcsec\squared} \Big] = 21.572 + \mathrm{M}_{\odot} - 2.5 \log_{10} \Big( \mu \Big[ \si{\Lsun\per\pc\squared} \Big] \Big) \quad .
  \end{equation}



  \item
  Die Oberflächenhelligkeit $\mu$ sollte man jedoch nicht mit der Oberflächendichte $\Sigma$ (engl.\ \textit{surface density}) verwechseln.
  Zwar sind die Einheiten gleich, die Berechning ist jedoch anders:
  \begin{equation}
    \Sigma \; \Big[ \si{\mag\per\arcsec\squared} \Big] = \frac{m_{\mathrm{tot}} \; [\si{\mag}]}{A \; \Big[ \si{\arcsec\squared} \Big]} \quad ,
  \end{equation}
  wobei $m_{\mathrm{tot}}$ die gesamte Helligkeit innerhalb der Fläche $A$ ist.
  Man summiert also zunächst den gesamten Fluss in $A$ und bestimmt daraus eine Helligkeit.
\end{enumerate}

\clearpage
\section{Konstanten}
\label{sec:konstanten}



\end{document}
