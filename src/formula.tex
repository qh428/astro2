\clearpage
\section{Formeln und Umrechnungen}
\label{sec:formeln_und_umrechnungen}

Dieser Abschnitt wird in zwei Unterkategorien aufgeteilt: Photometrie und Kinematik.
Zunächste wiederholen wir ein paar grundlegende Formeln und Umrechnungen aus der Vorlesung ``Einführung in die Astronomie 2''.


\subsection{Photometrie}
\label{subsec:photometrie}

\begin{enumerate}[label=$\smalltriangleright$]
  \item
  Der Strahlungsfluss $F$ (engl.\ \textit{flux}) gibt die eingehende Energie pro Fläche und Zeit an.
  Üblicherweise hat $F$ die Einheit [\si{\erg\per\centi\metre\squared\per\second}].

  \item
  Ein Schwarzer Strahler (engl.\ \textit{black body}) ist ein idealisierter Körper, welcher jede einfallende Strahlung absorbiert.
  Er emittiert Strahlung, die der Planck'schen Funktion (engl.\ \textit{Planck's law}) folgt:
  \begin{equation}
    B_{\nu, T} = 2 \Big( \frac{\nu}{\mathrm{c}} \Big)^{2} \frac{\mathrm{h}\nu}{\exp \big( \frac{\mathrm{h}\nu}{k_{\mathrm{B}} T} \big) - 1} \quad,
  \end{equation}
  mit den Variablen $\nu$ als Frequenz und $T$ als Temperatur.
  $\mathrm{h}$ ist das Planck'sche Wirkungsquantum, $\mathrm{c}$ die Lichtgeschwindigkeit, und $k_{\mathrm{B}}$ die Boltzmannkonstante.

  \item
  Der Strahlungsfluss $F$ eines Schwarzen Körpers hängt mit seiner Oberflächentemperatur $T$ zusammen.
  Es gilt
  \begin{equation*}
    F = \sigma_{\mathrm{SB}} T^{4} \quad ,
  \end{equation*}
  wobei $\sigma_{\mathrm{SB}}$ als Stefan-Boltzmann Konstante (engl.\ \textit{Stefan-Boltzmann constant}).

  In der Astronomie nehmen wir an, dass Sterne ideale Schwarze Körper sind.
  Dies bedeutet, dass wir die Sterne durch eine effektive Temperatur $T_{\mathrm{eff}}$ beschrieben werden können.
  Diese Temperatur besitzt ein Schwarzer Körper, wenn er den selben Strahlungsfluss wie ein Stern emittiert.
  Daher erhalten wir eine Relation zwischen dem Fluss eines Sterns und dessen effektive Temperatur:
  \begin{equation}
    F = \sigma_{\mathrm{SB}} T_{\mathrm{eff}}^{4} \quad  \quad .
  \end{equation}

  \item
  Das Wien'sche Verschiebungsgesetz (engl.\ \textit{Wien's displacement law}) ist eine Approximation für das Maximum der oben aufgeführten Planck'schen Funktion.
  Es ist gegeben durch
  \begin{align}
    \nu_{\mathrm{max}} \; [\si{\hertz}] & \approx 3 \frac{k_{\mathrm{B}} T}{\mathrm{h}} \\
                                     & \approx \frac{\SI{5.879e10}{\hertz\per\kelvin}}{T} \quad .
  \end{align}

  Man kann das Verschiebungsgesetz auch also Funktion der Wellenlänge $\lambda$ ausdrücken:
  \begin{equation}
    \lambda_{\mathrm{max}} \; [\si{\micro\metre}] \approx \frac{\SI{2898}{\micro\metre\per\kelvin}}{T} \quad .
  \end{equation}


  \item
  Die scheinbare Helligkeit (engl.\ \textit{apparent magnitude}) $m$ wird durch einen Strahlungsfluss $F$ bestimmt.
  Es gilt
  \begin{equation}
    m = -2.5 \log_{10} \bigg( \frac{F}{F_{\mathrm{ref}}} \bigg) \quad ,
  \end{equation}
  wobei $F_{\mathrm{ref}}$ ein Referenzfluss ist.
  Typischerweise nutzt man für $F_{\mathrm{ref}}$ den Fluss des Sterns Vega, jedoch existieren auch andere Referenzflüsse.

  \item
  Die Oberflächenhelligkeit $\mu$ (engl.\ \textit{surface brightness}) gibt den Fluss $F$ pro Fläche $A$ an:
  \begin{align}
    \mu \; [\si{\mag\per\arcsec\squared}] & = -2.5 \log_{10} \bigg( \frac{F}{F_{\mathrm{ref}}} \cdot \frac{1}{A [\si{\arcsec\squared}] } \bigg) \\
                                          & = m + 2.5 \log_{10} (A [\si{\arcsec\squared}]) \quad .
  \end{align}

  \item
  Die Oberflächenhelligkeit $\mu$ sollte man jedoch nicht mit der Oberflächendichte $\Sigma$ (engl.\ \textit{surface density}) verwechseln.
  Zwar sind die Einheiten gleich, die Berechning ist jedoch anders:
  \begin{equation}
    \Sigma = \frac{m_{\mathrm{tot}}}{A} \quad ,
  \end{equation}
  wobei $m_{\mathrm{tot}}$ die gesamte Magnitude innerhalb der Fläche $A$ ist.
  Man summiert also zunächst den gesamten Fluss in $A$ und bestimmt daraus eine Magnitude.
\end{enumerate}
