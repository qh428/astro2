\clearpage
\section{Notation}
\label{sec:notation}

Zunächst werden wir die in diesem Dokument verwendeten Symbole aufzählen.


\begin{ThreePartTable}
  \renewcommand\TPTminimum{\textwidth}

  %\begin{TableNotes}
  %  \item[(a)] My item
  %\end{TableNotes}

  \begin{longtable}{ l @{\extracolsep{\fill}} *{3}{l}}
    \caption[]{Notation für dieses Dokument}\\
    \label{tab:notation}\\
    \toprule
    \multicolumn{1}{c}{Begriff} & \multicolumn{1}{c}{Notation} & \multicolumn{1}{c}{Einheit} \\
    \midrule
    \endfirsthead

    \toprule
    \multicolumn{1}{c}{Begriff} & \multicolumn{1}{c}{Notation} & \multicolumn{1}{c}{Einheit} \\
    \midrule
    \endhead

    \midrule[\heavyrulewidth]
    \multicolumn{3}{l}{\textit{fortgesetzt}}\\
    \endfoot

    \midrule[\heavyrulewidth]
    %\insertTableNotes
    \endlastfoot

    \entry{Fluss}{flux}{$F$}{[\si{\erg\per\centi\metre\squared\per\second}]}
    \entry{Distanz}{distance}{$d$}{[\si{\centi\metre}] oder vergleichbar}
    \entry{Scheinbare Helligkeit}{apparent magnitude}{$m$}{[\si{\mag}]}
    \entry{Absolute Helligkeit}{absolute magnitude}{$M$}{[\si{\mag}]}
    \entry{Leuchtkraft}{luminosity}{$L$}{[\si{\erg\per\second}] oder vergleichbar}
    \entry{Distanzmodul}{distance modulus}{\dm}{[\si{\mag}]}
    \entry{Oberflächenheligkeit}{surface brightness}{$\mu$}{[\si{\mag\per\arcsec\squared}], [\si{\Lsun\per\pc\squared}]}
    \entry{Rotverschiebung}{redshift}{$z$}{N/A}

  \end{longtable}
\end{ThreePartTable}
